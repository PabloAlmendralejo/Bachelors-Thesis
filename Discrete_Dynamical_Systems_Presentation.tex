%----------------------------------------------------------------------------------------
%    PACKAGES AND THEMES
%----------------------------------------------------------------------------------------

\documentclass[aspectratio=169,xcolor=dvipsnames]{beamer}
\usetheme{Madrid}

\setbeamertemplate{footline}{
  \leavevmode%
  \hbox{%
    \begin{beamercolorbox}[wd=.3\paperwidth,ht=2.5ex,dp=1ex,left]{author in head/foot}%
      \usebeamerfont{author in head/foot} \hspace{1em} Pablo Borrego Ramos
    \end{beamercolorbox}%
    \begin{beamercolorbox}[wd=.4\paperwidth,ht=2.5ex,dp=1ex,center]{title in head/foot}%
      \usebeamerfont{title in head/foot} Sistemas Dinámicos Discretos en Dimensiones Bajas
    \end{beamercolorbox}%
    \begin{beamercolorbox}[wd=.3\paperwidth,ht=2.5ex,dp=1ex,right]{date in head/foot}%
      \usebeamerfont{date in head/foot} \insertshortdate \hspace{1em}  \hspace{1em}
    \end{beamercolorbox}%
  }%
  \vskip0pt%
}

\usepackage{hyperref}
\usepackage{graphicx} 
\usepackage{booktabs} 
\usepackage{amsmath}
\usepackage{amsthm}
%\documentclass[12pt,a4paper]{report}
\usepackage[utf8]{inputenc}
\usepackage[spanish]{babel}
\usepackage{amssymb}
\usepackage{amsthm}
\usepackage{enumitem}
\usepackage{amsmath}
\usepackage{amsthm}
\usepackage{amsfonts}
\usepackage{tikz}
\usepackage{wrapfig}
\usepackage{graphicx}
\usepackage{subcaption}
\usepackage{pdfpages}
\usepackage{pgfplots}
\usepackage{xcolor}
\usepackage{float}
%\usepackage[hidelinks]{hyperref}

\pgfplotsset{compat=1.18}

\newtheorem{definicion}{Definición}[section] 
\newtheorem{proposicion}{Proposición}[section] 
\newtheorem{teorema}{Teorema}[section] 


%----------------------------------------------------------------------------------------
%    TITLE PAGE
%----------------------------------------------------------------------------------------

\title{Sistemas Dinámicos Discretos en Dimensiones Bajas}

\author{Pablo Borrego Ramos}

\institute
{
    Facultad de Ciencias. PCEO Matemáticas y Estadística \\
    Universidad de Extremadura 
}
\date{\today} 

%----------------------------------------------------------------------------------------
%    PRESENTATION SLIDES
%----------------------------------------------------------------------------------------

\begin{document}

\begin{frame}
    \titlepage
\end{frame}

% \begin{frame}{Motivación y objetivos}
%    \tableofcontents
% \end{frame}

%------------------------------------------------
\section{First Section}
%------------------------------------------------
\begin{frame}{Motivación y Objetivos}

\textbf{Motivación}

Los sistemas dinámicos discretos unidimensionales son interesantes por su simplicidad estructural, pero además permiten modelar y analizar una gran variedad de fenómenos complejos, suelen estar asociados a ecuaciones en diferencias. Estos sistemas encuentran aplicaciones en diversas áreas como la biología,  la física y la criptografía. Un ejemplo clásico es la secuencia de Fibonacci, que ha sido utilizada para describir la evolución de poblaciones de conejos.

\vspace{0.5em}
\textbf{Objetivos}

\begin{itemize}
    \item \textbf{Capítulo 1:} Estudio de la Estabilidad en Sistemas Dinámicos Discretos
    \item \textbf{Capítulo 2:} Bifurcaciones en Sistemas Discretos Unidimensionales .
    \item \textbf{Capítulo 3:} Introducción al Caos
\end{itemize}

\end{frame}

%------------------------------------------------

\begin{frame}{Ecuaciones diferenciales y ecuaciones en diferencias}

    Una ecuación en diferencias es una expresión de la forma:
    \[
    G(n, x_n,x_{n+1}, \ldots, x_{n+k}) = 0 \quad \forall n \in \mathbb{N}
    \]
    Donde  $G:\mathbb{N} \times \mathbb{R}^{k+1}  \mapsto \mathbb{R}$ 

    \begin{block}{Solución de ecuación en diferencias}
    Denominaremos solución de la ecuación en diferencias a toda sucesión 
    $$
    x : \mathbb{N} \mapsto \mathbb{R}
    $$ 
    Que satisface la ecuación anterior. Llamaremos órbita a la imagen de las soluciones, es decir, al conjunto $\{x(n) \colon \, n \in \mathbb{N}\}$. 
    \end{block}

\end{frame}

%------------------------------------------------
\begin{frame}{Ecuaciones diferenciales explícitas en una dimensión}

    En este trabajo nos centraremos en las ecuaciones en diferencias donde $G$ tiene como dominio $\mathbb{N} \times \mathbb{R}^2$ y la ecuación en diferencias es de la forma:
    \[x_{n+1} = f(x_n),\quad \text{donde } f : \mathbb{R} \mapsto  \mathbb{R} \]

    Las soluciones de esta ecuación son sucesiones de la forma $\{x_0,f(x_0),f^2(x_0),\ldots \}$
\end{frame}

%------------------------------------------------

\begin{frame}{Métodos de discretización}
    Dada una ecuación diferencial $\dot{x}(t) = f(x(t))$ existen varios métodos que nos permiten convertir dicha ecuación en una ecuación en diferencias tomando una cantidad numerable de valores.

    \begin{block}{Método de Euler}
    El método de Euler discretiza la derivada de la función reemplazando \(\dot{x}(t)\) por el cociente de diferencias \((x_{n+1} - x_n)/h\). Así, la ecuación diferencial \(\dot{x} = f(x)\) se transforma en una ecuación en diferencias:
    \[
    x_{n+1} = x_n + h f(x_n).
    \]
    \end{block}



\end{frame}

%------------------------------------------------
\begin{frame}{Métodos de discretización}

    \begin{block}{Método de Newton}
    Definimos \( x_{n+1} \) como el punto donde la tangente a la gráfica de \( f(x) \) 
    en el punto \( (x_n, f(x_n)) \) corta el eje \( x \).
    La ecuación de la tangente es
    \[
    y - f(x_n) = f'(x_n)(x - x_n)
    \]
    Despejando el punto de corte con el eje $x$
    \[
    x_{n+1} = x_n - \frac{f(x_n)}{f'(x_n)}
    \]
    \end{block}    

\end{frame}

%------------------------------------------------

\begin{frame}{Dinámica Discreta de Funciones Escalares}

    \begin{definicion}[Punto Fijo]
    Dada una función $f: \mathbb{R} \rightarrow \mathbb{R}$ y un punto $\hat{x}$ diremos que $\hat{x}$ es un punto fijo cuando $f(\hat{x})=\hat{x}$. 
    \end{definicion}    
    \begin{definicion}
    Un punto fijo $\hat{x}$ de una función escalar $f$ es estable si para todo $\varepsilon > 0$ existe un $\delta > 0$ de tal forma que para todo $x_0$ tal que $|x_0 - \hat{x}| < \delta$  se cumple que $|f^n(x_0) - \hat{x}| < \varepsilon $ para todo $ n \in \mathbb{N}$. 
    Por el contrario, diremos que un punto fijo es inestable cuando no es estable.
    \end{definicion}
    

\end{frame}

%------------------------------------------------

\begin{frame}{Dinámica Discreta de Funciones Escalares}

    \begin{definicion}
        Un punto fijo \(\hat{x}\) del $f$ es un atractor global si para cada \(x_0 \in \mathbb{R}\), resulta
        \[
        \lim_{n \to \infty} x_n = \hat{x}
        \]
        Siendo \(x_n = f^n(x_0)\). En cambio, si solo ocurre para un entorno de $\hat{x}$, diremos que es un atractor local.
    \end{definicion} 
    
    \begin{definicion}
    
    Un punto fijo $\hat{x}$ de $f$ se dice globalmente (localmente) asintóticamente estable si satisface:
    \begin{itemize}
        \item $\hat{x}$ es estable.
        \item $\hat{x}$ es atractor global (local).
    \end{itemize}

    \end{definicion}



\end{frame}

%------------------------------------------------
    
\begin{frame}{Estabilidad en el caso hiperbólico}
    \begin{definicion}
    
    Dado un punto fijo $\hat{x} $ de una función escalar $f$ diremos que $\hat{x}$ es un punto hiperbólico cuando $|f'(\hat{x})| \neq 1$.
    \end{definicion}

    \begin{teorema}\label{Teorema1.3.1}
    Sea $f$ una función escalar de clase 1 y $\hat{x}$ un punto fijo hiperbólico de $f$.
    \begin{itemize}
        \item $\hat{x}$ es localmente asintoticamente estable $\iff$ $|f'(\hat{x})| <1$
        \item $\hat{x}$ es inestable $\iff$ $|f'(\hat{x})| > 1 $
    \end{itemize}
    \end{teorema}
\end{frame}

%------------------------------------------------

\begin{frame}{Demostración}

A través de una traslación, consideramos el punto $(0,0)$, basta definir la nueva función,
$$
g(u) = f(\hat{x} + u) - f(\hat{x})
$$
A continuación, definimos la máxima y mínima pendiente de la función en un entorno del punto fijo,
$$
m_\varepsilon =  \min_{|s| \leq \varepsilon}|f'(\hat{x}+s)|, \quad M_\varepsilon = \max_{|s| \leq \varepsilon}|f'(\hat{x}+s)|
$$
A través del Teorema del Valor Medio se puede probar que,
\[
m_\varepsilon |u| \leq |g(u)| \leq M_\varepsilon |u|
\]
Y aplicando inducción tenemos
$$
|u|  m^n_\varepsilon \leq |g^n(u)| \leq  |u|  M^n_\varepsilon \quad \forall n \in \{0,1 \ldots k\}
$$
    
\end{frame}

%------------------------------------------------

\begin{frame}{Demostración}

Cuando \( |f'(\hat{x})| < 1 \), la continuidad de \( f'(x) \) garantiza la existencia de un entorno de \( \hat{x} \) en el que \( |f'(\hat{x} + s)| < 1 \) y por tanto, en dicho entorno $M_\epsilon < 1$. Esto, junto con la desigualdad que demostramos previamente, nos permite ver que 
$$ 
|g(u)|  < |u|M_\epsilon < |u|
$$
%
Aplicando un argumento inductivo, concluimos que 
$$ 
|g^n(u)| < |u|M^n_\epsilon < |u|
$$ 
De aquí es fácil deducir la estabilidad y la estabilidad asintótica.


\end{frame}

%------------------------------------------------

\begin{frame}{Demostración}
Cuando \( |f'(\hat{x})| > 1 \) existe un $\delta > 0$ tal que si $|x - \hat{x}| < \delta$ entonces $|f'(\hat{x} + x)| > 1$.

\vspace{1em}

Para demostrar que $\hat{x}$ es inestable basta ver que existe un  $\hat{n} \in \mathbb{N}$ tal que $g^{\hat{n}}(u) \notin (-\delta,\delta)$ para cualquier $u \in (-\delta,\delta)$.

\vspace{1em}

Se demuestra por reducción al absurdo, suponemos que $g^n(u) \in (-\delta,\delta)$ para todo $n \in \mathbb{N}$ y todo $u \in (-\delta,\delta)$.
Como $|f'(\hat{x})| > 1$ tenemos que  $ \exists \delta > 0 \text{ tal que } m_\delta > 1$, teniendo en cuenta
\[
|u| m^n_\delta \leq |g^n(u)|
\]
%
Tendríamos entonces que $|u|m^n_{\delta}$ estaría acotada, lo cual es falso, pues como $m_\delta > 1$ $m^n_\delta \to \infty$ cuando $n \to \infty$.
\end{frame}

%------------------------------------------------

\begin{frame}{Estabilidad en el caso no hiperbólico}

\begin{teorema}\label{Teorema1.3.2}
Dada una función $f \in \mathcal{C}^3$ y un punto fijo $\hat{x} $ de $f$:

\begin{itemize}
    \item[1.] Si $f'(\hat{x}) = 1$, entonces tenemos tres casos a considerar:
    \begin{itemize}
        \item[(a)] Si $f''(\hat{x}) > 0$, entonces $\hat{x}$ es semi-asintóticamente estable por la izquierda.
        \item[(b)] Si $f''(\hat{x}) < 0$, entonces $\hat{x}$ es semi-asintóticamente estable por la derecha. 
        \item[(c)] Si $f''(\hat{x}) = 0$ pueden darse dos casos:
            \begin{itemize}
                \item[(c.1)] $f'''(\hat{x}) > 0$, entonces $\hat{x}$ es asintóticamente estable.
                \item[(c.2)] $f'''(\hat{x}) < 0$, entonces $\hat{x}$ es inestable.
            \end{itemize}
    \end{itemize}

    \item[2.] Si $f'(\hat{x}) = -1$ y $3 f''(\hat{x})^2 - 2 f'''(\hat{x}) \neq 0$, entonces tenemos dos casos a considerar:
    \begin{itemize}
        \item[(a)] Si $f'''(\hat{x}) + \frac{3}{2} f''(\hat{x})^2  < 0$, entonces $\hat{x}$ es asintóticamente estable.
        \item[(b)] Si $f'''(\hat{x})  + \frac{3}{2} f''(\hat{x})^2 > 0$, entonces $\hat{x}$ es inestable.
    \end{itemize}
\end{itemize}
\end{teorema}

\end{frame}

%------------------------------------------------

\begin{frame}{Bifurcaciones: Perturbación}

Dada una función $F$, diremos que una perturbación del sistema dinámico asociado a $f$ es una función de la forma:
$$
F : \mathbb{R}^k \times \mathbb{R} \rightarrow \mathbb{R} \quad  \quad (\lambda,x) \rightarrow F(\lambda,x) \quad \text{t.q}  \quad F(0,x)=f(x)
$$

\begin{definicion}
    Dada una función $f$ y una solución $\{x_0,x_1,x_2 , \ldots\} $ de $x_{n+1} = f(x_n)$ diremos que la solución es monótona creciente cuando $x_n \leq x_{n+1}$ $\forall n \in \mathbb{N}$ y decreciente cuando $x_n \geq x_{n+1}$ $\forall n \in \mathbb{N}$. Diremos que el sistema dinámico asociado a una función es monótono cuando toda órbita sea monótona. 
\end{definicion}
\end{frame}

%------------------------------------------------

\begin{frame}{Estabilidad de las bifurcaciones}

\begin{proposicion}
    
Supongamos que $f \in \mathcal{C}^1$ tiene un sistema dinámico monótono, $f(0)=0$ y $f'(0) \neq 1$. Consideramos la función $\mathcal{C}^1$ $F(\lambda,x)$ que cumple que $F(0,x) = f(x) $.
Entonces existe un entorno de $x = 0$ en el que para valores pequeños de $\mathbf{\lambda}$ $F(\lambda,x)$ tiene un único punto fijo con la misma estabilidad que la del punto fijo $0$ de $f$.

\end{proposicion}
    
\end{frame}

%------------------------------------------------

\begin{frame}{Puntos fijos degenerados de una función
cuadrática}

\begin{proposicion}

Considerando una función $f$ que cumple $f(0)=0, \; f'(0) =1,\; f''(0) \neq 0$ y una perturbación $F\in\mathcal{C}^2$. Sea $\psi(\lambda)$ tal que $H(\lambda, \psi(\lambda)) = 0$ donde $H(\lambda, x) := \frac{\partial (F(\lambda, x) - x)}{\partial x}$ y denotando $\alpha(\lambda) = F(\lambda,\psi(\lambda)) - \psi(\lambda)$, tenemos:

\begin{enumerate} 

\item Si $\alpha(\lambda) \cdot f''(0) < 0 $ $\Rightarrow  F$ tiene dos puntos fijos para valores pequeños de $\|\lambda\|$.

\item Si $\alpha(\lambda) \cdot f''(0) = 0 $ $\Rightarrow  F$ tiene un punto fijo para valores pequeños de $\|\lambda\|$.

\item Si $\alpha(\lambda) \cdot f''(0) > 0 $ $\Rightarrow  F$ no tiene puntos fijos para valores pequeños de $\|\lambda\|$.
\end{enumerate}
\end{proposicion}

\end{frame}

%------------------------------------------------

\begin{frame}{}



\begin{figure}[H]
    \centering
    \begin{subfigure}{0.45\textwidth}
        \centering
        \includegraphics[width=\textwidth]{fig47 (1).png}
        \label{fig:mi_imagen_9}
    \end{subfigure}
    \hfill % Ensures the figures are properly aligned
    \begin{subfigure}{0.45\textwidth}
        \centering
        \includegraphics[width=\textwidth]{fig48 (1).png}
        \label{fig:mi_imagen_10}
    \end{subfigure}
    \label{fig:mi_imagen_total1.8}
\end{figure}

 
\end{frame}

%------------------------------------------------

\begin{frame}{Bifurcaciones de doble periodo}

Diremos que una función \( f(x) \) cumple las hipótesis \( H_1 \) cuando: 
\begin{enumerate}
    \item  \( f(0) = 0 \) y \( f'(0) = -1 \).
    \item  \( (f \circ f)'''(0) \neq 0 \).
\end{enumerate}

Y dada una perturbación de un parámetro \( F_{\lambda}(x) \) de \( f(x) \), diremos que cumple las hipótesis \( H_2 \) cuando para todo $\lambda \in \mathbb{R}$:
\begin{enumerate}[start = 3]
    \item  \( F_{\lambda}(0) = 0 \).
    \item  \( \frac{dF_{\lambda}(0)}{dx} = -(1 + \lambda) \).
\end{enumerate}

\end{frame}

%------------------------------------------------

\begin{frame}{Bifurcaciones de doble periodo}

\begin{teorema}
Sea \( f(x) \in \mathcal{C}^3 \) una función con un punto fijo en \( 0 \) que cumple las hipótesis \( H_1 \) y dada una perturbación \( F_{\lambda}(x) \) de \( f(x) \) que cumple las hipótesis \( H_2 \), denotando $g = f \circ f$. Existe un entorno de \( (\lambda,x) = (0,0) \) en el cual la existencia de órbitas 2-periódicas viene determinada por \( g'''(0) \) y \( \lambda \) de la siguiente forma:

\begin{enumerate}
    \item  Para los valores de $\lambda$ tales que \( \lambda g'''(0) < 0 \) existe una única órbita  \( \{\hat{x}_\lambda, F(\lambda,\hat{x}_\lambda) \} \) tal que \( F(F(\lambda,\hat{x}_\lambda)) = \hat{x}_\lambda\). Además, las órbitas son asintóticamente estables (inestables) si \( 0 \) es un punto fijo inestable (asintóticamente estable) para ese valor de \( \lambda \).

    \item Cuando \( \lambda g'''(0) > 0 \), no existen órbitas periódicas tal que $F(F(\lambda,\hat{x}_\lambda)) = \hat{x}_\lambda$. 
\end{enumerate}

\end{teorema}

\end{frame}

%------------------------------------------------

\begin{frame}{Sistema Caótico}

\begin{definicion}

Decimos que el sistema dinámico discreto asociado a una función $ f : [\alpha, \beta] \mapsto [\alpha, \beta]$ es caótico si:

\begin{enumerate}
    \item Los puntos periódicos de $ f $ son densos en $ [\alpha, \beta] $.
    \item $ f $ es transitiva en $ [\alpha, \beta] $; es decir, dado cualquier par de subintervalos $ U_1 $ y $ U_2 $ en $ [\alpha, \beta] $, existe un punto $ x_0 \in U_1 $ y un $ n > 0 $ tal que $ f^n(x_0) \in U_2 $.
    \item $ f $ tiene dependencia sensible en $  [\alpha, \beta] $; es decir, existe una constante de sensibilidad $ \tau $ tal que, para cualquier $ x_0 \in [\alpha, \beta] $ y cualquier entorno de $ x_0 $, $ U $, existe algún $ y_0 \in U $ y $ n > 0 $ tal que:
    \[
    |f^n(x_0) - f^n(y_0)| > \tau
    \]
\end{enumerate}

\end{definicion}
    
\end{frame}

%------------------------------------------------

\begin{frame}{La ecuación logística}

La ecuación logística es una familia de funciones de la forma:
$$
f(\lambda, x) = \lambda x (1 - x) \quad \lambda > 1
$$
Tiene dos puntos fijos, uno en $0$ y otro en $\hat{x}_\lambda = 1 - \frac{1}{\lambda}$.

\vspace{1em}

Gracias a los teoremas anteriores, podemos deducir:
\begin{enumerate}
    
    \item Para $1 < \lambda < 3$ el $0$ es inestable y el $\hat{x}_\lambda = 1 - \frac{1}{\lambda}$ es asintoticamente estable.
    
    \item Para $\lambda = 3$ el $0$ es inestable y el $\hat{x}_\lambda = 1 - \frac{1}{\lambda}$ es estable.
    
    \item Para $ 3 < \lambda < 1 + \sqrt6 $ ambos puntos fijos son inestables
\end{enumerate}

\vspace{1em}

Y computacionalmente, podemos calcular:

\vspace{1em}

\begin{enumerate}

    \item Para $3.449 < \lambda < 3.570$ los valores de $\lambda$ dan comienzo a la aparición de unas 
    dinámicas muy complejas.

    \item Para  $\lambda = 3.839$ hay una única órbita periódica asintoticamente estable de periodo $3$.
    
\end{enumerate}

\end{frame}

%------------------------------------------------

\begin{frame}{La ecuación logística}

    \begin{figure}[H]
    \centering
    \includegraphics[width=0.5\textwidth]{fig41 (1).png}
    \caption{Podemos observar la presencia de caos en la órbita de  la ecuación logística cuando $\lambda> 4$}
    \label{fig:mi_imagen41}
    \end{figure}

\end{frame}

%------------------------------------------------

\begin{frame}{La ecuación logística}

\begin{proposicion}

Sea \(\Lambda\) el conjunto de puntos en \([0,1]\) con órbitas que nunca salen de \([0,1]\), entonces, $\Lambda$ es de la forma: 
\[
\Lambda=[0,1]-\bigcup_{n=0}^{\infty}A_{n}.
\]
%
Donde $A_0  = \{ x \in \mathbb{R} \text{ tal que } f_{\lambda}(x)>1 \}$ y los $A_n$ se definen de forma recursiva:
$$
A_n = f_{\lambda}^{-1}(A_{n-1}) \cap [0,1] \text{  para  } n \in \mathbb{N}
$$
\end{proposicion}
\end{frame}

%------------------------------------------------

\begin{frame}{Conjugaciones}

\begin{definicion}
    
Supongamos que $I$ y $J$ son intervalos y $f: I \to I$ y $g: J \to J$. Decimos que $f$ y $g$ son conjugadas si existe un homeomorfismo $h: I \to J$ tal que $h$ satisface la ecuación de conjugación $h \circ f = g \circ h$.

\end{definicion}

\begin{proposicion}\label{Prop5}
Supongamos que $f: I \to I$ y $g: J \to J$ son conjugadas vía $h$, donde tanto $I$ como $J$ son intervalos cerrados en $\mathbb{R}$ de longitud finita. Si el sistema dinámico discreto de $f$ es caótico en $I$, entonces el de $g$ también es caótico en $J$.
\end{proposicion}

\end{frame}

%------------------------------------------------

\begin{frame}{Dinámica Simbólica}

Consideremos una función $f : \mathbb{R}/\mathbb{Z} \mapsto \mathbb{R}/\mathbb{Z}$.

Dado un punto $x_0 \in  \mathbb{R}/\mathbb{Z}$, su órbita $x_{n+1} = f(x_n)$ y un recubrimiento de su imagen$\{I_0,I_1 \} $.


Sea \( \psi : \mathbb{R}/\mathbb{Z} \to \{0,1\}^{\mathbb{N}} \) la función que asigna a cada \( x \in \mathbb{R}/\mathbb{Z} \) una secuencia binaria, donde la coordenada \( n \)-ésima de \( \psi(x) \) viene dada por:
\[
\psi(x)_n = 
\begin{cases}
0, & \text{si } x_n \in I_0, \\
1, & \text{si } x_n \in I_1.
\end{cases}
\]

Sea \(\Sigma\) el conjunto de todas las secuencias posibles de $0$s y $1$s. Definimos la distancia \(d(s,t) : \Sigma \mapsto \Sigma\) simétrica, positiva y que satisface la desigualdad triangular.

\[
d(s,t)=\sum_{i=0}^{\infty}\frac{|s_{i}-t_{i}|}{2^{i}}.
\]


\end{frame}

%------------------------------------------------

\begin{frame}{La función de desplazamiento}

\begin{enumerate}
    \item \(\sigma\) tiene asociado un sistema dinámico discreto caótico.
    \item  \(\sigma\) es una conjugación discreta de \(f\) en \(\Lambda\).
    \item \(\sigma\) es completamente comprensible desde el punto de vista de los sistemas dinámicos.
\end{enumerate}


\begin{definicion}
    
Definimos la función de desplazamiento \(\sigma:\Sigma\rightarrow\Sigma\) como:
\[
\sigma(s_{0}s_{1}s_{2}...)=(s_{1}s_{2}s_{3}...)
\]
%
Esta función es continua, el conjunto de puntos periódicos de $\sigma$ es denso en $\Sigma$


\end{definicion}

\end{frame}

%------------------------------------------------

\begin{frame}{La ecuación Logística}

Definimos la función itinerario como la que asociará una secuencia infinita \(S(x_{0})=(s_{0}s_{1}s_{2}...)\) de 0s y 1s al punto \(x_{0}\) a través de la regla
\[
s_{j}=k \text{ si y solo si } f_{\lambda}^{j}(x_{0})\in I_{k}.
\]

\begin{teorema}

Cuando $\lambda > 4$ la función itinerario de $f_{\lambda}$,  $S : \Lambda \mapsto \Sigma$ es un homeomorfismo.

\end{teorema}


\end{frame}

%------------------------------------------------

\begin{frame}{La ecuación logística}

\begin{teorema}\label{Te3.3.3}

La función  \(S:\Lambda\rightarrow\Sigma\) proporciona una conjugación entre \(f_{\lambda}\) y la función de desplazamiento \(\sigma\).

\end{teorema}

Probamos que la función de desplazamiento $\sigma$ es caótica:

\begin{enumerate}
    \item densidad (sobre $\Sigma$): $S = \bigcup_{n=0}^\infty \{ (\overline{s_0,s_1,....,s_n}) $ $\text{ tal que } s_i = 0,1\}$ y la continuidad de $\sigma$
    \item transitividad: $s^{*}=(0100011011000001...)$ y órbita densa $\Rightarrow$ transitiva
    \item dependencia sensible: $s^{\prime} =(s_{0}s_{1}...s_{n}\hat{s}_{n+1}\hat{s}_{n+2}...)$ donde
    \(\hat{s}_{j}\) denota "no \(s_{j}\)" (es decir, si \(s_{j}=0\), entonces \(\hat{s}_{j}=1\), o si \(s_{j}=1\), entonces \(\hat{s}_{j}=0\)) cumple 
    \(d(s,s^{\prime})=1/2^{n}\) y $(d(\sigma^{n+1}(s),\sigma^{n+1}(s^{\prime})) = 2$
\end{enumerate}

\begin{teorema}
El sistema dinámico discreto asociado a la función $f_{\lambda}$ es caótico en $\Lambda$ cuando $\lambda > 4$.
\end{teorema}

\end{frame}

\end{document}
